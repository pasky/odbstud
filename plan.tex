\chapter{Future Work}
\label{ch:plan}

Ultimately, the goal of building a Question Answering system would
be a machine that can answer any kind of question a human could (or more).
But given the current state of art, I propose to consider this out of
scope as a doctoral thesis topic.  Instead, I shall propose two particular
real world QA scenarios that are inspired by real world applications.
Equipped with this vision and a baseline system of my own making,
I shall consider the scientific questions that we need
to answer in order to do well in these scenarios.  I want to propose
answering some of these questions to be the major focus of my thesis.

The first scenario involves answering \textbf{movie questions}
about metadata of films and TV series (credits, characters, dates,
awards, etc.).  This represents a proxy for answering questions about
product or company databases and many other business tasks.  The domain
is closed, but familiar to most users and with readily available knowledge
bases.  We already collected few hundred movie related questions as a dataset
and are working on YodaQA version for answering movie questions.%
\footnote{\texttt{d/movies} branch in the YodaQA Git repository.}

The second scenario aims to answer \textbf{prospectus questions}
about contents of an isolated document like a product page,
real estate offer or a usage manual.  Unlike the TODO \cite{MemNet},
the focus of this task is not answering logical riddles or following
through with a story, but rather entity recognition and relationship
extraction.

Both my proposals aim at realistic user interaction,
which brings in the first question ---
\textbf{How to link question keywords to appropriate concepts?}
That is, building a dataset and highly accurate solution for entity
linking (as surveyed in Sec.~\ref{sec:entitylink}.
Preliminary results on the movies dataset show that spelling mismatches
(both typos and interpunction variants) as well as disambiguation issues
(linking ``Hobbit'' to the best concept depending on question,
or identifying that ``the first time'' might not be a movie reference)
are non-trivial problems.
\cite{LeanFreebaseYao} agrees that the current
datasets like WebQuestions do not strain this issue realistically.
Some aspects (like fast fuzzy searches) are mostly of engineering interest,
but an example of semantically interesting scientific problem here
is a general mechanism to prefer certain classes of concepts based
on context, say movie instances of ``The Hobbit'' when asked about a director.
A baseline idea (slightly inspired by \cite{QuASE}) would be to rank concepts
based on similarity of vector embeddings (Sec.~\ref{sec:embeddings})
of the question and of the concept description.
We are already in progress of building an entity linking dataset to properly
compare various approaches.

\textbf{How to map unseen question focuses to knowledge graph edges?}
The existing structured QA proposals (as surveyed in Sec.~\ref{sec:structured})
rely on seeing each kind of question with many variations to infer explicit mappings
from in-question words to particular relations in the knowledge base ontology.
In open domain, the long tail of exotic questions would escape this;
in closed domain, amassing a large enough corpus of questions may be tricky.
At the same time, a human operator has no trouble finding the appropriate
relation based on its label.
Therefore, seeking a general way to match questions to labels of properties
(or even property paths) seems desirable, instead of simply learning a fixed
mapping from individual in-question words.
Again, leveraging an abstract representation of word meanings like vector
embeddings would be promising.

\textbf{How to pick the source sentences most likely to bear an answer?}
is an important question when dealing with question answering on unstructured
text corpora.  While Sec.~\ref{sec:anssentsel} surveys some candidates,
TODO (very problematic reference dataset; compound embeddings with NEs)

\textbf{How to integrate relation extraction approaches with question answering?}
is a research question that would boost the more naive approaches of answer
hypothesis generation by the research in relation extraction (TODO references).
One obvious approach is to first run general relation extraction on the knowledge
corpus, then query such a structured database.  However, I believe that the task
of ``reading'' text with a particular question in mind makes the task of identifying
pertaining relations considerably easier, and features used to extract a relation
would be useful for scoring the answers too.
Adapting well-known relation extraction strategies is a first step,
but going further and exploiting synergies with my work on the other questions,
I would like to experiment also with using embedded representation
of the sentences
(directly for relation extraction, or, say, as an attention mechanism
in the spirit of \cite{ReadAndComprehend}).

Many of the research questions might be in part answered by
vector embeddings of words, or whole sentences.
One reason we emphasize this approach so much is its high reusability potential:
a well developed solution to one problem might help with another one, both
on conceptual level and on the level of actual performance, e.g.\ applying
multi-task learning to learn a common representation of source artifacts.
To get there, we need to ask two important questions about vector embeddings
themselves, though.
\textbf{How to keep the signal level high and represent
complex statements when building compound embeddings?} points out that right
now, we do not know how to step from representing short and simple snippets or well known entities%
\footnote{When representing entities, we are thinking about representations
built in the spirit of \cite{QANTA}.}
to representing complex sentences (like in a typical Wikipedia article) or
rare entities.
Moreover, when we are to process sentences that contain numbers (like dates)
or references to entities, we need to answer
\textbf{How to embed or tie arbitrary data in addition to normal words
to compound embeddings?}
The current state of art does not seem to go beyond simple hacks like
carrying bag-of-tokens along the embeddings and counting intersect size
of that set when classifying relationships between two embedding \cite{Yu2014Deep}.
