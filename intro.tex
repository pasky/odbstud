\chapter{Introduction}
\label{ch:intro}

TODO some nice quote

In this report, I would like to propose a doctoral thesis to write
and defend at the Czech Technical University on the topic of question
answering systems.

This thesis proposal is a little unusual since I have radically changed
the topic of my research in the course of my second year of study:
from portfolio-based function optimization to question answering systems.
Therefore, the bulk of my research results published up to now are not
on topic of the proposed thesis; my work on portfolio-based optimization
is summarized in App.~\ref{app:opt}.

The current focus of my doctoral research
is improving the state-of-art in the field of factoid question answering.
My main results so far have been
building an extensive question answering system \textbf{YodaQA} \citep{YodaQAPoster2015}
and proposing a high-quality dataset \citep{YodaQACLEF2015},
but I have also applied the system to a biological QA domain \citep{YodaQABioASQ2015}
and achieved some new results not published yet (described later in the proposal).
This research is primarily supervised by Dr. Jan Šedivý;
some of the newest results have been achieved while collaborating
with intern students in our group.

\section{Factoid Question Answering}

Let us consider the Question Answering problem --- a function of
unstructured user query that returns the information queried for.
This is a harder problem than a linked data graph search (which requires
a precisely structured user query) or a generic search engine (which
returns whole documents or sets of passages instead of the specific
information).

The Question Answering task is however a natural extension of a search
engine, as currently employed e.g.\ in Google Search \citep{googleKG}
or personal assistants like Apple Siri, and with the high
profile IBM Watson Jeopardy! matches \citep{WatsonOverview}
it has became a benchmark of progress in AI research.

My goal is ultimately building a general purpose QA system.
Thus, I consider an ``open domain'' general factoid question answering,
rather than domain-specific applications, though keeping flexibility
in this direction is certainly worthwhile.

TODO some examples of questions and answers

\section{Task Outline}

Diverse QA system architectures have been proposed in the last 15 years,
applying different approaches to information retrieval.  A full survey
follows, for now let me outline at least the most basic choices I faced
when designing my system.

First, the restriction to \textbf{factoid} questions means the system
is essentially an extension of a search engine (rather than deducing
facts logically) and should return precisely specified, relatively short
text snippets as short answers.  Answering happens mainly based on fact
lookup.  This is in contrast with different question answering tasks
(like Language Comprehension Entrance Exams) where the system needs to
process a text passage and answer tricky questions about the text meaning
(see Sec.~\ref{sec:nonfactoid}).

Perhaps the most popular approach in factoid QA research has been restricting
the task to querying structured knowledge bases, typically using the
RDF paradigm and accessible via SPARQL\@.  The QA problem can
be then rephrased as learning a function translating free-text user query
to a structured lambda expression or SPARQL query. \citep{Semantic2013Berant, Semantic2014Bordes}
I prefer to focus on unstructured datasets as the coverage of the system
as well as domain versatility increases dramatically; building a combined
portfolio of structured and unstructured knowledge bases
is then again an obvious extension.

When relying on unstructured knowledge bases, a common strategy is to offload
the information retrieval on an external high-quality web search engine
like Google or Bing (see e.g.\ the \textbf{AskMSR} system \citep{AskMSR}
or many others).
I make a point of relying purely on local information sources.%
\footnote{However, as non-benchmarked extension, YodaQA also includes
	Bing search as described in Sec.~\ref{sec:bing}.}
While the task becomes noticeably harder,
I believe the outcome is a more
universal system that could be readily refocused on a specific domain
or proprietary knowledge base, and also a system more appropriate as
a scientific platform as the results are fully reproducible over time.

Finally, a common restriction of the QA problem concerns only selecting
the most relevant answer-bearing passage, given a tuple of input question
and set of pre-selected candidate passages \citep{WangQAGrammar}.
This Answer Sentence Selection task is certainly
worthwhile as a component of a QA system but does not form a full-fledged
system by itself.
It may be argued that returning a whole passage is more useful for the user than a direct narrow answer,
but this precludes any reasoning or other indirect answer synthesis on the part of the system,
while the context and supporting evidence can be still provided by the user interface.
Direct answer output may be also used in a more general AI reasoning engine,
an idea that I keep in sight within my design though it is clearly
out of scope for the thesis I propose.

To summarize, the system I propose should produce short, clear answers
based on information retrieval from both
unstructured (full-text) and structured (database) knowledge bases,
and not rely on any ``omniscient web search'' to make the task easier
specifically in the open domain setting.

\section{This Thesis Proposal}

The rest of this proposal shall focus on building the case for
a thesis on the topic of factoid question answering.
Chapter~\ref{ch:survey}
surveys the field in detail, examining different formulations of the problem
as well as a variety of sub-tasks, reference datasets and approaches,
and recent progress in the field.
Chapter~\ref{ch:work} describes
the work I have done on the thesis so far --- this revolves mainly around
the question answering system YodaQA that I have built, but also progress
on some of the QA sub-tasks.
Chapter~\ref{ch:plan} outlines some of the specific problems of QA
to focus on in the thesis.
I conclude the proposal with a summary in Chapter~\ref{ch:concl}.
